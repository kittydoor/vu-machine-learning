\section{Background Info}

Machine Learning is a branch of Artificial intelligence based on the idea that systems can learn from data, identify patterns and make decisions using those patterns.
Machine Learning uses algorithms to "learn" from a given  (training) sample of data.
Then you use another sample (test sample) to determine the accuracy of the trained algorithms.
The higher the accuracy, the better the model will preform to more and more real life data.

One branch of Machine Learning (ML) is the Neural Networks (NN).
Neural networks are biologically inspired ML programs.
They are modelled after the neurons in the brains of animals.
As to how the NN works, a neuron (artificial though) receives the inputs and starts some operations on the input.
Next, the neuron sends the output to the other neurons (they have to be connected).
This system has three layers: Input Layer, Hidden Layers, and Output Layer.
In the first layer the training observations are given to the neurons.
In the Hidden Layers, which are located between input- and output layers, the system tries to analysis the relationships involved in data.
In these layers system will weight the nodes that are found to be more predictive of the outcome more heavily.

There are many types of Neural Networks.
One specific type is the Perceptron.
It is a mathematical model of the biological neuron.
This model uses an algorithm for binary classifiers, as well as a type of linear classifier (it makes it's prediction based on a linear model, combining a set of weights with the feature vector).
Normal Perceptrons use linear activation functions.
When they use three or more layers, the Perceptrion is called a Multilayer Perceptron (MLP).
In this case, each node is a neuron that uses nonlinear activation functions.
The MLP uses the technique backpropogation for training,
This technique calculates the gradient that is needed to calculate the weights which are used in the network.
The nonlinear activation functions in MLP usually follow the shape of the sigmoid function.
This is a mathematical function (logistical) with an "S" shaped curve, which has the following formulation:

\begin{equation*}
  \centering
  S(x)=\frac{1}{1+e^{-x}}
\end{equation*}


Mean Squared Error measures the average of squares of the errors.
The errors are defined as the difference between the real value of data points and the estimated value.
This measurement shows the quality of the regression.
Still there are many others measurements which can be used for quality of the regression and the estimated values.
Mean Percentage Error,  Mean Absolute Percentage Error and Root Mean Squared Error are some of the other common quality-measurements.
Let $\hat(Y)$ be a vector of $n$ predictions, and $Y$ is the vector of real values then the MSE is as follows:

\begin{equation*}
  MSE= \frac{1}{n} \sum_{i=1}^{n}(Y_{i}-\hat{Y}_{i})^{2}
\end{equation*}
